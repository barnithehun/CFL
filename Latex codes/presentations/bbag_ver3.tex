%----------------------------------------------------------------------------------------
%	PACKAGES AND THEMES
%----------------------------------------------------------------------------------------

% !TEX program = pdflatex
% !TEX enableSynctex = true
% !BIB program = bibtex

\documentclass[notes]{beamer}

\usetheme{Frankfurt}

%-- Layout --%
% font size
\usepackage[fontsize=11pt]{scrextend}
\usepackage{amsmath}
\usepackage{appendixnumberbeamer} 
\usepackage[short]{optidef}
\usepackage[backend=bibtex,style=authoryear]{biblatex}
% page
\setbeamersize{text margin left=1cm, text margin right=1cm}
\setbeamercolor{button}{bg=blue, fg=white}
% line space 
% footline
\addtobeamertemplate{navigation symbols}{}{
    \usebeamerfont{footline}
    % \usebeamercolor[fg]{footline}
    \hspace{2em}
    \insertframenumber/\inserttotalframenumber    }
% headline
\setbeamercovered{transparent}

% for the indicator function
\usepackage{dsfont}

%-- Figure --%
\usepackage{graphicx}
% \usepackage{subfig}

\DeclareMathOperator*{\argmin}{arg\,min} % Jan Hlavacek
\DeclareMathOperator*{\argmax}{arg\,max} % Jan Hlavacek


%-- Table --%
\usepackage{tabularx,booktabs}
\newcolumntype{Y}{>{\centering\arraybackslash}X}

%-- Text --%
\usepackage{xcolor}
% \usepackage{tikz}

%-- Math --%
\usepackage{amssymb}
\usepackage{amsmath}

%-- Caption --%
\usepackage{caption}
\usepackage{subcaption}

%-- Effects --%
\usepackage{hyperref}

%----------------------------------------------------------------------------------------
%	TITLE PAGE
%----------------------------------------------------------------------------------------

\title[]{The Nature of Corporate Credit Constraints \\
\small Update on the theory} % The short title appears at the bottom of every slide, the full title is only on the title page

\author{Barnab\'as Sz\'ekely} % Your name

\date{\today } % Date, can be changed to a custom date

\begin{document}
\renewcommand{\arraystretch}{1.4}

\begin{frame}
\titlepage % Print the title page as the first slide
\end{frame}

%%----------------------------------------------------------------------------------------
%	PRESENTATION SLIDES
%---------------------------------------------------------------------------------


\section{Introduction, literature}
%------------------------------------------------
\begin{frame}[label=1] 
\frametitle{Credit market frictions in macro-finance}
Traditional approach:
\begin{itemize} \setlength\itemsep{0em} \small
    \item Asset based borrowing constraints \\
    \textit{Kiyotaki and Moore (1997); Bernanke, Gertler and Gilchrist (1999)}   
    \item ABCs in the misallocation literature \\ \textit{Midrigan and Xu (2014), Banerjee and Moll (2010), Buera, Kaboski and Shin (2011)}
\end{itemize} \normalsize \vspace{5mm} \\

Prevalence of earnings based constraints:
\begin{itemize} \setlength\itemsep{0em} \small
    \item Lian and Ma (2021): 80\% of US corporate debt is CF-based
    \item Drechsel (2023): 3 out of the 4 most often used debt covenants are earnings-based
\end{itemize} \normalsize

\end{frame}


\begin{frame}[label=1] 
\frametitle{Credit market frictions in macro-finance}

Response to shocks: 
\begin{itemize}\setlength\itemsep{0em} \small
    \item  TFP and `investments ' shocks have different effects depending on the debt contract in place \\ \textit{Drechsel (2023); Choi (2022)}
    \item  Financial accelerator of BGG is mitigated \\ \textit{Lian and Ma (2021)}
\end{itemize} \normalsize

Monetary policy: 
\begin{itemize} \setlength\itemsep{0em} \small
    \item Sticky prices interact with EBCs, but not with ABCs \\ \textit{Drechsel (2023), Öztürk (2023), Greenwald (2019)}
\end{itemize} \normalsize

Macroprudential policy: 
\begin{itemize} \setlength\itemsep{0em} \small
    \item Firms do not over-borrow (rather, under-borrow) \\ \textit{Drechsel and Kim (2022)}
\end{itemize} \normalsize


\end{frame}

\begin{frame}[label=slide2] \frametitle{Data and Motivation}
Compustat and Capital IQ data; US traded firms; 2010-2023
\begin{itemize}  \setlength\itemsep{0em}
    \item Most firms typically hold multiple debt contracts (5.21 on average) and use on both type of debt!
\end{itemize} \vspace{0.1mm} \\
\begin{table}[h!]
\centering
\resizebox{0.95\textwidth}{!}{% 0.75 times the text width
\begin{tabular}{l|rrr|r}
\multicolumn{5}{l}{\textbf{Firm shares by debt financing strategy}} \\
\toprule
\textbf{Share by} & \textbf{\# of Firms} & \textbf{Assets}  & \textbf{Borrowing} & \textbf{Constraint} \\  
Asset-based firms & 35.66\% & 4.83\% & 3.36\% & ABCs \\ 
'Hybrid' firms & 52.11\% & 87.27\% & 90.31\% & In-between \\ 
CF-based firms & 12.23\% & 7.91\% & 6.33\% & EBCs \\ 
\bottomrule
\end{tabular}
}
\label{tab:shares}
\end{table} 
\end{frame}

\section{Alternative approach}

%------------------------------------------------
\begin{frame}[label=1] 
\frametitle{Asset based vs. CF-based debt contracts}
Asset-based debt contracts
\begin{itemize}
 \setlength\itemsep{0em}
    \item Backed by collateral 
    \item In default: recovering collateral
    \item Comparative advantage: under liquidation
\end{itemize} \vspace{5mm} \\

Earnings-based debt contracts
\begin{itemize}
 \setlength\itemsep{0em}
    \item Backed by continuation value 
    \item In default: recovering the continuation value
    \item Comparative advantage: under reorganization
\end{itemize} 

\end{frame}

%------------------------------------------------
\begin{frame}[label=2] \frametitle{The tradeoff}
\begin{table}[h!]
\centering
\resizebox{0.9\textwidth}{!}{% 0.75 times the text width
\begin{tabular}{l|cc}
   & Liquidation  & Reorganization \\  
  \midrule
 Asset based contract & $ \phi_a (1-\delta) k$  &  $ \phi_a (1-\delta) k$ \\
 CF based contract & $\phi_c (1-\delta) k$ & $ \phi_v V(x,\varepsilon) - \zeta $ \\ 
\bottomrule
\end{tabular}
}
\label{tab:shares}
\end{table}  
 $\phi_a > \phi_c \rightarrow $ AB contracts are better at recovering collateral

\end{frame}

%------------------------------------------------
\begin{frame}[label=3] \frametitle{Liquidation decision}
Household / Courts makes the liquidation decision: 
\begin{itemize}  \setlength\itemsep{0em}
    \item Decision is optimal:
 $$ \max \Big\{ \phi_v V(k,b,\varepsilon)- \zeta, \  \phi_k (1-\delta) k \Big\}.$$
    \item No strategic thinking on the firm / lender side 
    \item Revenues and expenses appear in the HH budget constraint
\end{itemize}  \vspace{5mm} \\ 
Ex-ante probability of liquidation:
$$ \gamma(k',b', \varepsilon) = E \left[ \mathds{1}(\phi_k (1-\delta) k  \  \geq \ \phi_v V(k,b,\varepsilon)- \zeta)  \right] $$


\end{frame}


%------------------------------------------------
\begin{frame} \frametitle{External finance premium}
The firm chooses the CFL reliance, $\tau$ to maximize $q(k',b',\varepsilon,\tau)$ \vspace{5mm} \\ 
From lenders' zero profit condition: 
\footnotesize
\begin{align*}
    & q(k',b',\varepsilon, \tau)b' =   \beta \left[ (1-P_\chi) b' +  \\
         & \hspace{3mm} P_\chi \gamma(k',b', \varepsilon) \left[ \min\{b', \ (1-\tau) \phi_A (1-\delta) k' +\tau \phi_C (1-\delta) k' \} \right] + \\
         & \hspace{3mm} P_\chi (1-\gamma(k',b', \varepsilon)) \left[ \min\Big{\{}b', \ (1-\tau) \phi_A (1-\delta) k' +\tau \left( \phi_v E \left[V^D_1(k',b',\varepsilon')\right] - \zeta \right) \Big{\}} \right] \right]  
\end{align*} \vspace{0.1mm} \\ 
\normalsize
Optimal debt financing strategy can be isolated:
\small
$$ \tau^*(k',b',\varepsilon) = \argmax_\tau q(k',b', \varepsilon, \tau) \hspace{10mm} s.t: \tau \in (0,1) $$

\end{frame}


%------------------------------------------------
\begin{frame}
\frametitle{Defaults and firm values}
Value of a continuing firm: 
\begin{equation*} \label{eq:V_2}
V(x,\varepsilon) = \max_{k',b'} \left(x - k' +  q(k',b',\varepsilon)b' + (1-P_\chi)\beta E \left[ V(x',\varepsilon') \right] \right)
\end{equation*} 
subject to: 
\small
\begin{align*}
    & q(k',b',\varepsilon) =   \frac{\beta}{b'}\left[ (1-P_\chi) b' +  \\
         & \hspace{3mm} P_\chi \gamma \left[ \min\{b', \ (1-\tau^*) \phi_A (1-\delta) k' +\tau^* \phi_C (1-\delta) k' \} \right] + \\
         & \hspace{3mm} P_\chi (1-\gamma) \left[ \min\Big{\{}b', \ (1-\tau^*) \phi_A (1-\delta) k' +\tau^* \left( \phi_v E \left[V^D_1(k',b',\varepsilon')\right] - \zeta \right) \Big{\}} \right] \right]  
\end{align*}
$$ x' = \pi(k',\varepsilon')+(1-\delta)k'-b' $$
$$ x - k' +  q^s(k',b',\varepsilon)b' \geq 0 $$
\normalsize


\end{frame}

\section{Conclusion}

%------------------------------------------------
\begin{frame}\frametitle{Conclusion}
Advantages:
\begin{itemize}
 \setlength\itemsep{0em}
    \item More intuitive, easier preliminary solutions
    \item CFL reliance as a result of optimization (not mechanical from the probability of default)
\end{itemize} \vspace{5mm} \\

Drawbacks: 
\begin{itemize}
 \setlength\itemsep{0em}
    \item Liquidation decision must be done by a 3rd party
    \item Revenues and payoff accounting is more complicated
\end{itemize} 

\end{frame}

\appendix

\section{Appendix}

%------------------------------------------------
\begin{frame}\frametitle{Accounting payoffs - Liquidation}
Firm exits - capital is shared between the lender and the HH
\begin{itemize}  \setlength\itemsep{0em}
    \item Cost of liquidation: $(1-\phi_A)(1-\delta)k$
    \item The lender receives:
    \small  $$ \min \{ b, \hspace{3mm} (1-\tau) \phi_A (1-\delta) k + \tau \phi_C (1-\delta) k \} $$ \normalsize
    \item  The household receives whatever is left after paying the firm:
    \small  $$ \max \{\phi_A (1-\delta) k - b, \hspace{3mm} \tau (\phi_A-\phi_C) k  \} $$ \normalsize
    \end{itemize}
 HH payoff is needed only to make accounting consistent
\end{frame}

%------------------------------------------------
\begin{frame}\frametitle{Accounting payoffs - Reorganization}
Firm continues - the  HH pays to keep the firm going
\begin{itemize} \setlength\itemsep{0em}
    \item The lender gets a share in the firm after its CF-based debt
    \item Resells it to the household at price $\tau V$
    \item The total cost of reorganization is: $ (1- \phi_v) V + \zeta $ 
    \item It is shared between the HH and lender (in proportion of $\tau$)
\end{itemize}
The household is willing to keep the firm going if: 
$$\phi_k (1-\delta) k  \  < \ \phi_v V(k,b,\varepsilon)- \zeta$$

\end{frame}

%------------------------------------------------
\begin{frame}\frametitle{Accounting payoffs - Reorganization}
\textbf{The HH pays: }
\begin{itemize} \setlength\itemsep{0em}
    \item Capital value after asset-based debt: $(1-\tau)\phi_A(1-\delta)k$
    \item Firm value after CF-based debt: $\tau V$
    \item His share of the reorganization cost: $ (1-\tau) [(1- \phi_v) V + \zeta)]$ 
\end{itemize}
Pays in total: $$ \max \{b - [(1- \phi_v) V + \zeta)], (1-\tau)\phi_A(1-\delta)k + \tau V + (1-\tau) [(1- \phi_v) V + \zeta)]\}  $$

\textbf{The firm,}
\begin{itemize} \setlength\itemsep{0em}
    \item pays:  $ \tau [(1- \phi_v) V + \zeta)]$
    \item receives: $(1-\tau)\phi_A(1-\delta)k + \tau V $
\end{itemize}
Receives in total: $$ \min \{b , (1-\tau)\phi_A(1-\delta)k + \tau (\phi_v V - \zeta) \}  $$

\end{frame}


\end{document}
